\documentclass[12pt,a4paper]{article}
%-----------------------PACKAGES-----------------------%
\usepackage[top=1in,bottom=1in,left=0.5in,right=0.5in]{geometry}
\usepackage{graphicx}
\usepackage{array}
\usepackage{xcolor}
\usepackage{adjustbox}
\usepackage{titlesec}
\usepackage{svg}
\usepackage{lettrine}

%-----------------------TABLES ALIGNEMNET-----------------------%

%-----------------------TITLE DOCUMENT-----------------------%
\begin{document}
	\begin{Titlepage}
\begin{center}
    \vspace*{2cm}
    
    \textbf{\Huge Expert-Staging signup\\Expert Admin sms apis\\Expert front Admin sms\\}
    \vspace*{2cm}
    
      
      \vspace{0.2cm}
      \begin{center}
          \large M.Wahaj Tahir\\wahajtahir01@gmail.com\\wahajt@acm.org
      \end{center} 
    
    \vspace{1.5cm}
    \begin{center}
    \large October,2023 
    \end{center}
    
    \vfill
    \vspace{0.8cm}
    \begin{figure}[hb]
        \centering
        \includegraphics[scale=0.50]{Logo/Website logo.png}
    \end{figure}
    \end{center}
\end{Titlepage}









 
 \clearpage
%-----------------------TABLE OF CONTENT-----------------------% 
\tableofcontents
\clearpage
%-----------------------TABLE OF CONTENT-----------------------% 
\section{Introduction}
The Service Inventory module is designed to manage various types of services, providing a centralized system for storing and organizing services with different attributes and characteristics. This module will offer a range of features to support the efficient management of services.
\subsection{Purpose}
The purpose of this document is to outline the functional and non-functional requirements of the Service Inventory module, defining its capabilities and expected behavior.

\section{Functional Requirement}
\begin{table}[h]
\caption{Functional Requirements}
    \centering
    \begin{tabular}{|l|p{10cm}|}
    \hline
       \textbf{Functionality No.}&\textbf{Description} \\ %end of row
       \hline
       \textbf{FR-1}&The system must allow administrators to associate multiple prices with each service based on the unit of measurement. Prices can vary based on different attributes.\\ %end of row
       \hline
         \textbf{FR-2}&Services can be defined with dynamic attributes, which can be customized based on the specific characteristics of the service.\\ %end of row
       \hline
       \textbf{FR-3}&Administrators can attach various supporting documents or resources related to a service. \\ %end of row
       \hline
     
       \textbf{FR-4}&Services can be grouped into packages, providing a convenient way to bundle related services together.\\ %end of row
       \hline
       \textbf{FR-5}&Administrators can create service groups based on the nature of services. Services included in these groups can be automatically assembled into packages.\\ %end of row
       \hline
        \textbf{FR-6}&Services may have variable prices that can be updated by administrators as needed.\\ %end of row
       \hline
        \textbf{FR-7}&Each service can have a unique booking flow, allowing administrators to customize the service booking process.\\ %end of row
       \hline
        \textbf{FR-8}& Every service should have an associated cost, which can be defined and tracked. Costs can include expenses related to materials, labor, or other resources used in delivering the service.\\ %end of row
       \hline
    \end{tabular}
\end{table}
\newpage
\section{Non-Functional Requirements}
\begin{table}[h!]
\caption{Non-Functional Requirements}
    \centering
    \begin{tabular}{|l|p{7cm}|}
    \hline
       \textbf{Functionality No.}&\textbf{Description} \\ %end of row
       \hline
       \textbf{FR-1}&The module shall implement robust security measures to protect sensitive data and ensure only authorized users can make changes.\\ %end of row
       \hline
         \textbf{FR-2}&The system must be able to handle a large number of services and their associated data efficiently, ensuring minimal latency.
\\ %end of row
       \hline
       \textbf{FR-3}&The user interface must be intuitive and user-friendly to facilitate easy management of services and related features.
 \\ %end of row
       \hline
    \end{tabular}
\end{table}

\end{document}