\documentclass[12pt,a4paper]{article}
%-----------------------PACKAGES-----------------------%
\usepackage[top=1in,bottom=1in,left=0.5in,right=0.5in]{geometry}
\usepackage{graphicx}
\usepackage{array}
\usepackage{xcolor}
\usepackage{adjustbox}
\usepackage{titlesec}
\usepackage{svg}
\usepackage{lettrine}
\usepackage{lettrine}
\usepackage{svg}
\usepackage{caption}
\usepackage{blindtext}
\usepackage{fancyhdr}
\usepackage{hyperref}
\hypersetup{
    colorlinks=true,
    linkcolor=blue,
    filecolor=magenta,      
    urlcolor=cyan,
    pdftitle={Overleaf Example},
    pdfpagemode=FullScreen,
    }


%-----------------------TABLES ALIGNEMNET-----------------------%

%-----------------------TITLE DOCUMENT-----------------------%
\begin{document}
	\begin{Titlepage}
\begin{center}
    \vspace*{2cm}
    
    \textbf{\Huge Expert-Staging signup\\Expert Admin sms apis\\Expert front Admin sms\\}
    \vspace*{2cm}
    
      
      \vspace{0.2cm}
      \begin{center}
          \large M.Wahaj Tahir\\wahajtahir01@gmail.com\\wahajt@acm.org
      \end{center} 
    
    \vspace{1.5cm}
    \begin{center}
    \large October,2023 
    \end{center}
    
    \vfill
    \vspace{0.8cm}
    \begin{figure}[hb]
        \centering
        \includegraphics[scale=0.50]{Logo/Website logo.png}
    \end{figure}
    \end{center}
\end{Titlepage}









 
 \clearpage
%-----------------------TABLE OF CONTENT-----------------------% 
 \tableofcontents
 \clearpage
%-----------------------TABLE OF CONTENT-----------------------% 
\section{Schedule Management}
\subsection{Scope}
In schedule management module the provider will be able to set the schedule slots and can make the slot repeat week wise,how ever he/she wants.
\subsection{Functional Requirement}
\begin{table}[h!]
\caption{Functional Requirements}
    \centering
    \begin{tabular}{|l|p{7cm}|}
    \hline
       \textbf{Functionality No.}&\textbf{Description} \\ %end of row
       \hline
       \textbf{FR-1}&The provider should be able to create schedule slots specifying date, start time, end time, and any additional relevant information (e.g., location or service type).\\ %end of row
       \hline
         \textbf{FR-2}&The provider should have the option to make slots repeat on a weekly basis. This should include defining the recurrence pattern (e.g., every Monday and Wednesday) and the end date for recurring slots.\\ %end of row
       \hline
       \textbf{FR-3}& The system should maintain and display the availability of slots in real-time, ensuring that overbooking or double-booking is prevented.\\ %end of row
       \hline
     
       \textbf{FR-4}& Providers should be able to edit or delete existing slots as needed. Modifications should be reflected accurately in the schedule.\\ %end of row
       \hline
       \textbf{FR-5}&The provider can schedule multiple slots within multiple different  business.\\ %end of row
       \hline
    \end{tabular}
\end{table}
\newpage
\subsection{Non-Functional Requirement}
\begin{table}[h!]
\caption{Non-Functional Requirements}
    \centering
    \begin{tabular}{|l|p{7cm}|}
    \hline
       \textbf{Non-Functionality No.}&\textbf{Description} \\ %end of row
       \hline
       \textbf{NFR-1}&The system should respond swiftly to user actions, even when dealing with a large number of schedules and recurring slots. \\ %end of row
       \hline
       \textbf{NFR-2}&The system should respond swiftly to user actions, even when dealing with a large number of schedules and recurring slots.\\ %end of row
       \hline
       \textbf{NFR-3}&The system should be scalable to accommodate a growing number of providers and clients without compromising performance.\\ %end of row
       \hline
       \textbf{NFR-4}&The schedule management module should have a high level of availability and reliability. It should be operational 24/7 with minimal downtime.
        \\ %end of row
       \hline
       \textbf{NFR-5}&The user interface should be intuitive and user-friendly, requiring minimal training for providers to create, manage, and understand their schedules.
       \\ %end of row
       \hline
       % \textbf{NFR-6}&
       %  \\ %end of row
       % \hline
       % \textbf{NFR-7}&
       %  \\ %end of row
       % \hline
       % \textbf{NFR-8}&
       % \\ %end of row
       % \hline
       % \textbf{NFR-9}&
       %  \\ %end of row
       % \hline
       % \textbf{NFR-10}&
       % \\ %end of row
       % \hline   
    \end{tabular}
\end{table}


\end{document}